\subsection{Cross Product}

Another important product that is defined in $\bbr^3$ is the \textbf{Cross Product}

\begin{definition}{Definition of Cross Product}
Given two vectors $\vec{v}$ and $\vec{w}$ in $3$-dimensional Euclidean space, write their \emph{cross product} as $\vec{v}\times \vec{w}$.  To compute it, if we have $$\vec{v}=\begin{bmatrix} v_1 \\ v_2 \\  v_3\end{bmatrix}$$ and $$\vec{w}=\begin{bmatrix} w_1 \\ w_2 \\ w_3\end{bmatrix},$$ then $$\vec{v}\times \vec{w}=\begin{bmatrix} v_2w_3-v_3w_2 \\ v_3w_1-v_1w_3 \\ v_1w_2-v_2w_1 \end{bmatrix}.$$ 
\end{definition}

One of the most efficient ways to compute the cross product is by way of a linear algebra algorithm known as the \textbf{determinant}. \hypertarget{det}{So let's talk about determinants}. The determinant of a square matrix is essentially a scalar summary of the matrix.

\begin{definition}{Determinant of a $2\times 2$ Matrix}
Let $$M=\bmat{a&b\\c&d}. $$ Then the \textbf{determinant} of $M$, $\det(M)$ is:
$$\det(M)=\vmat{a&b\\c&d}=ad-bc. $$
\end{definition}

To do determinants of larger matrices, we use a technique called \textit{cofactor expansion}.

\begin{definition}{Determinant of a $3\times 3$ Matrix}
Let $$ M=\bmat{a&b&c\\d&e&f\\g&h&i}. $$ Then the \textbf{determinant} of $M$, $\det(M)$ is:
\begin{align*}
\det(M)=&\vmat{a&b&c\\d&e&f\\g&h&i}\\
=&a\vmat{e&f\\h&i}-b\vmat{d&f\\g&i}+c\vmat{d&e\\g&h}\\
=&a(ei-fh)-b(di-fg)+c(dh-eg).
\end{align*}
\end{definition}

\begin{exercise}{Determinants}
Find the determinant of the matrix $$\bmat{1&-1&2\\-3&2&1\\6&1&-5}.$$
\end{exercise}

In fact, we can use the determinant to calculate the cross product.

\begin{claim}{Cross Product as a Determinant}
Let $$\vcv=\bmat{v_1\\v_2\\v_3} \text{ and } \vcw=\bmat{w_1\\w_2\\w_3}.$$ Then $$\vcv\times\vcw=\vmat{\vci & \vcj & \vck \\ v_1 & v_2 & v_3 \\ w_1 & w_2 & w_3} $$
where $\vci, \vcj, \vck$ are the unit orthonormal vectors in 3-space, that is, $$\vci=\bmat{1\\0\\0}, \ \vcj=\bmat{0\\1\\0}, \text{ and } \vck=\bmat{0\\0\\1}. $$
\end{claim}

\begin{exercise}{}
\begin{enumerate}
\item Let $\vcv=\bmat{3\\-1\\5}$ and $\vcw=\bmat{0\\4\\2}$, Find $\vcv\times\vcw$.
\vspace{1em}
\item Let $\vcv=\vci+6\vcj-8\vck$ and $\vcw=4\vci-2\vcj-\vck$. Find $\vcv\times\vcw$.

\end{enumerate}
\end{exercise}

\begin{definition}{\hypertarget{crossprops}{Properties of the Cross Product}}
\begin{itemize}
\item The cross product is anticommutative. That is, $\vcv\times\vcw=-\vcw\times\vcv$.
\vspace{1em}
\item The cross product is \textit{not} associative. That is, in general, $\vcv\times(\vcu\times\vcw)\neq(\vcv\times\vcu)\times\vcw$.
\vspace{1em}
\item The cross product distributes over vector addition on both the left and the right. That is, $\vcv\times(\vcu+\vcw)=\vcv\times\vcu+\vcv\times\vcw$ and $(\vcu+\vcw)\times\vcv=\vcu\times\vcv+\vcw\times\vcv$.
\vspace{1em}
\item The cross product $\vcv\times\vcw$ is orthagonal to both $\vcv$ and $\vcw$.
\vspace{1em}
\item The magnitude of the cross product $||\vcv\times\vcw||$ is equal to the area of the parallelogram generated by $\vcv$ and $\vcw$.
\vspace{1em}
\end{itemize}
For proofs of these facts and more, refer to our Vector Operations Properties overleaf collaborative project.
\end{definition}

Note also that the cross product, while not defined in $\bbr^2$ can still be computed over $\bbr^2$ by adding an extra, zero coordinate to each vector and computing the cross product in $\bbr^3$.

\begin{example}{Cross in $\bbr^2$}
Let $\vcv=\bmat{2\\-3}$ and let $\vcu=\bmat{-1\\4}$. Let's move these two vectors from $\bbr^2$ into $\bbr^3$ by padding in an extra zero in the third coordinate. So then $$\vcv=\bmat{2\\-3\\0},\ \vcu=\bmat{-1\\4\\0}. $$
Then we can compute $\vcv\times\vcu$
\begin{align*}
\vcv\times\vcu=&\bmat{2\\-3\\0}\times\bmat{-1\\4\\0}\\
=&\bmat{-3\cdot0-0\cdot4\\0\cdot(-1)-2\cdot0\\2\cdot4-(-3)\cdot(-1)}\\
=&\bmat{0\\0\\5}.
\end{align*}
Note that the resulting cross product should always orthogonal to both vectors, so certainly it should \textit{always} be a vector that lies on the $z$-axis.
\end{example}

Also, just like with the dot product, the cross product has a relationship to the angle between two vectors.

\begin{theorem}{Cross Products and Angles Between Vectors}
Let $\vcv$, $\vcu$ be 2 or 3 dimensional vectors. Let $\theta$ be the smallest angle between $\vcv$ and $\vcu$. Then $$||\vcv\times\vcu||=||\vcv||\cdot||\vcu||\sin(\theta). $$
\end{theorem}

\begin{exercise}{}
Use the \textbf{cross} product to find the following angles.
\begin{enumerate}
\item Let $\vcv=2\vci+2\vcj$ and let $\vcu=-\vci+\vcj$. Find the angle between $\vcv$ and $\vcu$.
\vspace{1em}
% \item Let $\vcv=\bmat{3\\2\\1}$ and let $\vcu=\bmat{2\\-5\\1}$. Find the angle between $\vcv$ and $\vcu$. Note this is different than the angle you got when you did this with dot product! This has to do with sine's inverse being annoying. The correct angle is from the dot product. This one gives the supplement of the correct angle.
\vspace{1em}
\end{enumerate}
\end{exercise}