\subsection{Partial Derivatives and Tangent Planes}
Of course, this being a Calculus class, we probably want to get to some Calculus. So we start, as usual, with derivatives. However, we immediately run into a problem. When taking derivative of some function $y=f(x)$, we are interested in the derivative with respect to $x$: $f'(x)=\frac{dy}{dx}$. But what is $f'(x,y)$? Is it $\frac{d}{dx}f(x,y)$? What even does $\frac{d}{dx}f(x,y)$ even mean? Certainly, $y$ isn't a constant, but also it doesn't make any sense to use implicit differentiation, since $y$ isn't necessarily related to $x$ in some specific way. It turns out that in order to make sense of the situation, we must define a new type of derivative. Don't panic though, it's easier than it sounds! Basically, the \textbf{partial derivative} is a special type of derivative for surfaces that allows us to hold all other variables constant.

\begin{definition}{Partial Derivatives}
\begin{itemize}
\item The \textbf{partial derivative} of $f(x,y)$ with respect to $x$ is defined as the derivative of $f(x,y)$ with respect to $x$, while holding $y$ constant. It is written as $f_x$ or $\frac{\del f}{\del x}$. The partial differential operator with respect to $x$ is $\delx{}$. 
\vspace{1em}
\item The \textbf{partial derivative} of $f(x,y)$ with respect to $y$ is defined as the derivative of $f(x,y)$ with respect to $y$, while holding $x$ constant. It is written as $f_y$ or $\frac{\del f}{\del y}$. The partial differential operator with respect to $y$ is $\dely{}$. 
\end{itemize}
\end{definition}

\begin{exercise}{Partial Derivatives}
Consider $f(x,y)=\sqrt{16-x^2-y^2}$.
\vspace{1em}
\begin{enumerate}
\item Find $f_x$, the partial of $f$ with respect to $x$.
\vspace{1em}
\item Find $f_y$, the partial of $f$ with respect to $y$.
\end{enumerate}
\end{exercise}

The partial derivative acts somewhat like the regular derivative from Calculus I, but deals with an issue inherent to the situation-- the ``slope" or rate of change on a surface depends entirely on which direction you are going along. The partial with respect to $x$ tells us the ``slope" (or rate of change of $z$ with respect to $x$) as we move parallel to the $x$-axis, while the partial with respect to $y$ tells us the ``slope" (or rate of change of $z$ with respect to $y$) as we move parallel to the $y$-axis. These two ``slopes" come together to generate the equation of the \textbf{tangent plane}.

Note that when talking about planes in this section, we will not be using standard form, exactly. Instead, we will be presenting planes as some $z=f(x,y)$. So then, the plane $ax+by+cz=d$ would be written as $$f(x,y)=\frac{d}{c}-\frac{a}{c}x-\frac{b}{c}y.$$

\begin{definition}{Tangent Plane}
Let $f(x,y)$ be a surface. Then the plane tangent to $f(x,y)$ at the point $\big(x_0,y_0,f(x_0,y_0)\big)$ is $$\ell(x,y)=f(x_0,y_0)+f_x(x_0,y_0)(x-x_0)+f_y(x_0,y_0)(y-y_0). $$
\end{definition}

\begin{exercise}{Tangent Plane Equation}
Consider a function $f(x,y)$ and a point on that surface, $\big(x_0,y_0,f(x_0,y_0)\big)$. Then the partial with respect to $x$ at $(x_0,y_0)$ is $f_x(x_0,y_0)$ and the partial with respect to $y$ at $(x_0,y_0)$ is $f_y(x_0,y_0)$.
\vspace{1em}
\begin{enumerate}
\item Explain why the vectors $$\vcf_x=\bmat{1\\0\\f_x(x_0,y_0)}\text{ and } \vcf_y=\bmat{0\\1\\f_y(x_0,y_0)}$$
are parallel to the tangent plane to $f(x,y)$ at $(x_0,y_0)$.
\vspace{1em}
\item Using those two vectors and the point $\big(x_0,y_0,f(x_0,y_0)\big)$, give the equation of the tangent plane in vector dot product form, $$a(x-x_0)+b(y-y_0)+c(z-z_0)=0. $$
\item Solve the above equation of the plane for $z$ and show that you get the equation of the tangent line from the definition above.
\end{enumerate}
\end{exercise}

\begin{exercise}{Tangent Planes}
Find the equation of the plane tangent to $f(x,y)=\sqrt{16-x^2-y^2}$ at the point $(1,1)$.
\end{exercise}

We can, of course, take multiple partial derivatives in the same way that we can take multiple regular derivatives. However, this brings up a new possibility of \textit{mixed} partial derivatives.

\begin{definition}{Multiple Partial Derivatives}
Let $z=f(x,y)$ be an infinitely differentiable surface.
\vspace{1em}
\begin{itemize}
\item The second partial derivative with respect to $x$ is defined as $$f_{xx}(x,y)=\delx{}\big(f_x(x,y)\big). $$
\item The second partial derivative with respect to $y$ is defined as $$f_{yy}(x,y)=\dely{}\big(f_y(x,y)\big). $$
\item The mixed partial derivatives are defined as
$$f_{xy}(x,y)=\dely{}\big(f_x(x,y)\big),\text{ and }f_{yx}(x,y)=\delx{}\big(f_y(x,y)\big),$$ depending on whether we take the $x$ partial or the $y$ partial first.
\end{itemize}
\end{definition}

In general, the partial differential operators of different variables do not commute. However, in practice the two mixed partials are \textit{often} equal. The conditions for this are given by the following theorem of Clairaut and Schwarz,
 (Clairaut being one of the first to prove it and Schwarz being the first rigorous proof).
 
 \begin{theorem}{\hypertarget{cs}{Clairaut-Schwarz Theorem}}
Let $f(x,y)$ be a function $f:D\to\bbr$ where $D$ (the domain) is a subset of $\bbr^2$. Let $(x_0,y_0)$ be in $D$. Then, if some neighborhood of $(x_0,y_0)$ is contained in $D$ and $f_{xy}$ and $f_{yx}$ are continuous on that neighborhood, $$f_{xy}(x_0,y_0)=f_{yx}(x_0,y_0). $$
 \end{theorem}

 Note: A neighborhood is a way of saying some collection of all the points near another point. You can think of it like an open disk (i.e. all the points that are within $\varepsilon$ of $(x_0,y_0)$,) but a neighborhood need not be a regular shape. We could just state a stronger version of this theorem with an open rectangle or open disk, but I think neighborhood is a very useful and intuitive topographical term!
 
 Regardless, since most of the functions we see are continuous almost everywhere and infinitely differentiable almost everywhere, the \textbf{mixed partials are almost always equal.}
 
 \begin{exercise}{An Example of Clairaut-Schwarz Working!}
 Consider the function $$f(x,y)=\sin(x^2y^3).$$
 \begin{enumerate}
 \item Find $f_{xy}(x,y)$.
 \vspace{1em}
 \item Find $f_{yx}(x,y)$.
 \vspace{1em}
 \item Verify the criteria for Clairaut-Schwarz. That is, is $f(x,y)$ continuous? Are the mixed partials continuous?
 \vspace{1em}
 \item Verify the result for Clairaut-Schwarz. Are the two mixed partials equal?
 \end{enumerate}
 \end{exercise}
