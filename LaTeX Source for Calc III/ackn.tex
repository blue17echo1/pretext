\section*{Notes About Notes}
Welcome to Calculus III! These notes and practice problems are intended to form a large proportion of the coursework for this course. This text is by no means a \textit{complete} reference on multivariable Calculus. Instead, the text is meant as a specific companion for our specific course, without extra overhead. If you would like an outside reference, my suggestion would to be either to pick up an old edition of Stewart's Calculus from a used books seller (this should be pretty cheap!), or to use \href{https://tutorial.math.lamar.edu/Classes/CalcIII/CalcIII.aspx}{Paul's Online Math Notes}.

A few notes on the notes:

\begin{itemize}
\item The text is designed around a course in 3 parts. 

\begin{enumerate}
\item The first part is chapters 1, 2 and 3, which cover an introduction to vectors and Calculus on functions that have vector inputs.
\item The second part is chapters 4 and 5, which cover partial derivatives and multiple integrals, or Calculus on functions that have vector outputs.
\item The third part is chapters 6 and 7, which cover line and surface integrals, or Calculus on functions with vector inputs and outputs.
\end{enumerate}

Each part has an associated exam review. Each chapter also includes a section with homework and miscellaneous practice, labeled with an R.

\item This text uses hyperlinks throughout. Some hyperlinks are internal (the table of contents links to the respective sections) and some are external (some exercises and examples have links to Geogebra for visualization). Unfortunately, this functionality is at least partially lost in a paper version of the text if you choose to print it out.
\item The text includes both examples and exercises. Any exercise that requires interaction is denoted with a slightly darker coverplate. For example:

\begin{example}{An Example of an Example}
Examples, definitions, theorems and claims look like this!
\end{example}

\begin{exercise}{An Example of an Exercise}
Exercises look like this!
\end{exercise}

\item \hypertarget{PON} Some exercises or examples are adapted from \href{https://tutorial.math.lamar.edu/Classes/CalcIII/CalcIII.aspx}{Paul's Online Math Notes}. When an exercise is adapted this way, it will be denoted with a small P in the title, as below:

\begin{pexercise}{An Example of an Exercise, Adapted}
This example would have been adapted from \href{https://tutorial.math.lamar.edu/Classes/CalcIII/CalcIII.aspx}{Paul's Online Math Notes}. 
\end{pexercise}

\item These notes are OER (open education resources) and should be available for free. You can access the \href{https://www.overleaf.com/read/qyqfjbgpysqx}{LaTeX source code here}.

%\item The notes are currently typeset in the \href{https://brailleinstitute.org/freefont}{Atkinson Hyperlegible Font} from the Braille Institute. The intention is to enhance readability for readers with low or impaired vision.
\end{itemize}

If you have any questions, concerns or comments, please reach out to me by email!

Aaron Allen, Front Range Community College
\begin{verbatim}
aaron.allen@frontrange.edu
\end{verbatim}